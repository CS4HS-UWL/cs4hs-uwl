% Typeset with LaTeX format
% Using AMS extensions for mathematical formatting

%	The 'myarticle' class gives me a smaller title size
%	This class should appear as 'article'
%	in any submitted source file, so that others can
%	properly format it.  Please change if it is not correct.
%  	I use the 'leqno' option, for left-handed equation numbers.
\documentclass[leqno,11pt]{article}
%	To get all the formal devices, I usually use
%	all of 'amsmath', 'amssymb', and 'amsthm'.
%	There may also appear the packages
%	'setspace' (for double-spacing)
%	and 'mylogic' (for personal margins, headers, commands, etc.)
%	These should NOT appear in any submitted source file.
%	Please remove them if still here upon submission.
\usepackage{amsmath,amssymb,amsthm,amscd,amsxtra,latexsym}
\usepackage{notation}
\usepackage{enumerate}
\usepackage{graphicx}
\usepackage{color}
\usepackage{hyperref}

%---------------------------------------------
%       OPTIONAL:  PAGE SPACING HERE
%---------------------------------------------

%\usepackage{setspace}
%	The following line may be commented out
% 	If so, it can be deleted without affecting anything.
%	I use the command (with the 'setspace.sty' package)
%	for alternative formatting (also \doublespacing).
%	If this line is commented out, and 'setspace' is not in use,
%	the document formats as usual for its class.
%%\onehalfspacing

%---------------------------------------------
%       PAGE FORMAT (BORDERS/HEADERS)
%---------------------------------------------

%	Page format commands:
%	Override normal article margins,
%	making the margins smaller
\setlength{\textwidth}{6.5in}
\setlength{\textheight}{8.5in}
\setlength{\oddsidemargin}{0in}
\setlength{\evensidemargin}{0in}
\setlength{\topmargin}{0in}

%	More format:
\thispagestyle{empty}

%---------------------------------------------
%       DEFINED COMMANDS
%---------------------------------------------

%---------------------------------------------
%       DOC STARTS HERE
%---------------------------------------------

\begin{document}

\begin{center}
{\large
	\textbf{Reinforcement Learning Exercise}
}

\hrulefill
\end{center}

\noindent
\textbf{Instructions:}  This exercise involves implementing a very simple reinforcement learning algorithm to
learn a net-positive strategy in a simple card game.  While the player will start off playing randomly, and
poorly, they will quickly converge on a strategy that has positive overall expectation.


\begin{enumerate}
	\item \textbf{Setting Up}: Two decks of cards are filled with an equal number of face cards (Jack,
		Queen, King, Ace), where cards can repeat multiple times.  The two decks are shuffled, and
		placed side by side.  (Code is supplied that will generate a simulation of this process and
		display cards.) Each player is given a copy of the worksheet.  The player should fill in each of
		the boxes in the top two columns with an initial $Q$-value of $0$.
		
	\item On each round of the game, the players turn over a card from the left-hand deck. They then choose
		an action (either \texttt{Higher} or \texttt{Lower}).  This choice is handled as follows:
		\begin{itemize}
			\item Each player compares the two values in the row for the face-value of the card seen.
				If one value is greater than the other, the corresponding action is chosen.
				For instance, if the card is an \texttt{Ace} and the values under
				\texttt{Higher} and \texttt{Lower} are $1$ and $0$, respectively, the player
				chooses \texttt{Higher}.  (Note: these numbers can be positive or negative.)
			\item If the two values are the same, then the player tosses a coin: if the coin comes
				up Heads, they choose \texttt{Higher}, and else they choose \texttt{Lower}.
		\end{itemize}
		(The player may want to write down their choice.)

	\item After the players choose their actions, the next card in the right-hand (opponent) deck is turned
		over.  Scoring is handled as follows:
		\begin{itemize}
			\item If the player has chosen \texttt{Higher}, then:
				\begin{itemize}
					\item If the opponent card is higher than the player card, the player
						wins.  They record a score of $+1$ in the next unfilled
						box at the bottom of the sheet, and add $1$ to the
						$Q$-value for \texttt{Higher} in the row 
						with the face-value of the \textbf{player's card}.
					\item If the opponent card is lower than the player card, the player
						loses.  They record $-1$ in the next unfilled
						box at the bottom of the sheet,  and subtract $1$ from the
						$Q$-value for \texttt{Higher} in the row associated
						with the face-value of the \textbf{player's card}.
					\item If the two cards are the same, this is a tie.  The player records
						a $0$ in the next unfilled box at the bottom of the sheet.  No
						$Q$-value updates are needed.
				\end{itemize}
			\item If the player has chosen \texttt{Lower}, then:
				\begin{itemize}
					\item If the opponent card is lower than the player card, the player
						wins.  They record a score of $+1$ in the next unfilled
						box at the bottom of the sheet, and add $1$ to the
						$Q$-value for \texttt{Lower} in the row 
						with the face-value of the \textbf{player's card}.
					\item If the opponent card is higher than the player card, the player
						loses.  They record $-1$ in the next unfilled
						box at the bottom of the sheet,  and subtract $1$ from the
						$Q$-value for \texttt{Lower} in the row associated
						with the face-value of the \textbf{player's card}.
					\item If the two cards are the same, this is a tie.  The player records
						a $0$ in the next unfilled box at the bottom of the sheet.  No
						$Q$-value updates are needed.
				\end{itemize}
	
		\end{itemize}
\end{enumerate} 

\end{document}
 
